%Version en ligne
\documentclass[12pt,a4paper]{report}
\newcommand{\caml}{\textsc{Caml}}
%Version imprimable
%\documentclass[12pt,a4paper,twoside,openright]{report}

% ***** relative path from this file's folder to the template folder *****
\newcommand{\baserelpath}{../../../..} % FIXME

% ***** preamble (needed, do not remove or modify) *****
\newcommand{\baserelTPpath}{\baserelpath/templates/tp}
% ************************
% * fichier de préambule *
% ************************

% ***** extensions *****
\usepackage[T1]{fontenc}	% Caractères accentués et césure
\usepackage[utf8x]{inputenc}	% Choix de l'encodage
\usepackage{graphicx}		% Paquage pour l'insertion d'images
\usepackage{wrapfig}		% Figures flottantes
\usepackage[french]{babel}	% Support Français
\usepackage{eurosym}		% Signe € qui s'adapte en fonction de la police
\usepackage{fancyhdr}		% En-tête de pages personalisés
\usepackage{color}		% Un peu de couleurs
\usepackage{array}		% Faire des beaux tableaux
% Liens cliquables dans le document
\usepackage[colorlinks=true,linkcolor=black,urlcolor=blue,%
pdftitle={Workshop GConfs},pdfauthor={GConfs}]{hyperref}
\usepackage{amsmath}		% Formules matématiques
\usepackage{listings}           % Mise en forme de code source

% Changement des polices du document
\usepackage{palatino}
\usepackage[sc]{mathpazo}
\linespread{1.05}


% ***** césures particulières *****

\hyphenation{}

% ***** filigrane *****
%\usepackage{eso-pic,rotating}
%\AddToShipoutPicture{%
%\unitlength 1cm
%\put(11,15){%
%\begin{rotate}{45}
%\makebox(0,0){\color{lightgray}\scalebox{3.5}{\Huge BROUILLON}}
%\end{rotate}}
%}

% ***** couleurs perso *****

\definecolor{lightgray}{RGB}{240,240,240}

\definecolor{vert}{RGB}{0,176,80}
\definecolor{verta}{RGB}{79,97,40}
\definecolor{vertb}{RGB}{195,214,155}
\definecolor{vertc}{RGB}{235,241,221}
\definecolor{bleua}{RGB}{31,73,125}
\definecolor{bleub}{RGB}{219,229,241}
\definecolor{rougea}{RGB}{192,80,77}
\definecolor{rougeb}{RGB}{242,220,219}

\definecolor{brown}{RGB}{128,0,0}
\definecolor{blue}{RGB}{0,0,255}
\definecolor{green}{RGB}{0,128,0}
\definecolor{seagreen}{RGB}{69,158,181}


% ***** paramètre de la coloration syntaxique *****
\lstset{language=[Sharp]C,basicstyle=\ttfamily\footnotesize,%
stringstyle=\ttfamily\color{brown},commentstyle=\color{green},%
%numberstyle=\footnotesize,stepnumber=1,numbersep=7pt,%
backgroundcolor=\color{vertc},frame=single,tabsize=2,breaklines=true,%
breakatwhitespace=false,%
classoffset=0,% Mots clés non reconnus
morekeywords={},keywordstyle=\color{blue},%
classoffset=1,% Noms de classes
morekeywords={Console},keywordstyle=\color{seagreen},%
classoffset=0
}

% ***** commandes personnelles *****

\newcommand{\helpbox}[3][bleu]{\vspace{1em}\fcolorbox{#1a}{#1b}{\parbox{.92%
\linewidth}{\hspace{5pt}\textbf{#2~:} #3}}\vspace{0.75em}}
\newcommand{\warnbox}[3][rouge]{\vspace{1em}\fcolorbox{#1a}{#1b}{\parbox{.92%
\linewidth}{\hspace{5pt}\textbf{#2~:} #3}}\vspace{0.75em}}
\newcommand{\tp}{workshop}
\newcommand{\Tp}{Workshop}
\newcommand{\svn}{\textsc{svn}}
\makeatletter
\renewcommand{\chapter}{\clearpage%
     \@startsection{chapter}{1}{-0.75em}{\baselineskip}%
     {0.5\baselineskip}{\LARGE\textbf}}
\makeatother
\renewcommand{\thechapter}{\Roman{chapter}}
\renewcommand{\thesection}{\arabic{section}}

% ***** en-têtes et pieds de pages *****
\pagestyle{fancyplain}
	\lhead[\emph{\nouppercase{\leftmark}}]{\emph{\textit{\Tp{} TP\_title (in preambule.tex)}}}
	\chead{} 
	\rhead[\emph{\textit{\Tp{} TP\_title (in preambule.tex)}}]{\emph{\nouppercase{\leftmark}}}
	\lfoot[]{\small{\textit{GConfs 2011}}}
	\rfoot[\small{\textit{GConfs 2011}}]{}


% ***** for code listings *****
\lstsetparams{Sharp}{C}{}{Value,Convert,Exp,Cst}

% ***** workshop title (used as pages' head) *****
\newcommand{\workshoptitle}{CAS et évaluation d'expressions} % FIXME

% ***** page heading and foot *****
\headfootparams{\workshoptitle}

% ***** to display a "BROUILLON" in all pages *****
%\makefiligrane


\begin{document}

\title{
  \vspace{1cm}
  \textbf{\Huge{\Tp{} \workshoptitle}}\\
  %\vspace{1cm}
  %\includegraphics[scale=0.75]{images/logo.png}
}
\author{
  \Large{Sébastien \textit{AzrYel} \textsc{Crozet} ({\ttfamily crozet\_s})}\\\\
}

\date{
  \vspace{1cm}
  % GConfs logo
  \includegraphics[scale=0.5]{\baserelTPpath/images/gconfs.png}\\
  \vspace{0.5cm}
  Day xx Month 201x
}
\maketitle
\newpage
\tableofcontents
\newpage

% beginning of document
\section{Introduction}
\subsection{Environnement de Travail}
\begin{itemize}
\item Lancer Visual Studio 2008
\item Fichier / Nouveau / Projet / C\# / Application console
\item Une portion de code minimale sera générée:
\end{itemize}
\begin{lstlisting}
using System;
using System.Collections.Generic;
using System.Linq;
using System.Text;

namespace ConsoleApplication1
{
    class Program
    {
        static void Main(string[] args)
        { 
        }
    }
}
\end{lstlisting}		
Vous venez de mettre en place votre environnement de travail. Il est composé de plusieurs fichiers.
Parmi ceux-ci, seul un nous intéresse pour le moment: Program.cs\\
De nombreux autres fichiers seront générés tels que des fichiers contenant des informations de débogage.\\
Un fichier est toutefois important: ControleApplication1.sln. Il s'agit du fichier que vous devrez ouvrir si vous fermez puis ré-ouvrez VS2008. En effet,
il contient toutes les information concernant votre projet. (En réalité, il représente une "solution", et vous trouverez aussi un fichier ConsoleApplication1.csproj qui représente un "projet".
De ce fait vous pouvez avoir plusieurs "projets" dans une solution, mais ce n'est pas important pour ce TP).\\
\\
Comme vous pouvez le voir, le fichier Program.cs contient une fonction "Main". Il s'agit du point d'entrée de l'application: la première fonction appelée automatiquement lors du démarrage de votre programme.\\
C'est donc ici que tout commence et que vous ferez des appels à d'autres fonctions. Par la suite, nous allons écrire d'autres fonctions que vous devrez tester. Vous les écrirez dans ce fichier, mais gardez en tête qu'il
est parfois préférable de travailler dans plusieurs fichiers séparés.
\subsection{Hello World !}
Comme tout bon tutoriel qui se respecte, la première chose à vous faire
faire est d’afficher Hello World!. Pour cela il vous faudra utiliser la procédure
Console.WriteLine( ). Tous les paramètres donnés seront
convertis en chaîne de caractères (pour les types prédéfinis) puis
affichés sur la sortie standard.
Dans un premier temps nous allons le placer dans le fichier Program.cs, et compléter la fonction Main(string[] args).
\begin{lstlisting}
using System;
using System.Collections.Generic;
using System.Linq;
using System.Text;

namespace ConsoleApplication1
{
    class Program
    {
        static void Main(string[] args)
        {
            Console.WriteLine("Hello world");
            Console.ReadLine();
        }
    }
}

\end{lstlisting}
\emph{- Remarque }La fonction ReadLine() va attendre que l'utilisateur appuie sur
Entrée. Cela permet que le programme ne s'arrête pas directement mais nous
laisse le temps de lire notre Hello World!
\subsection{Voir le résultat de votre travail!}
Coder c'est bien, mais ça ne sert à rien si on ne peut pas exécuter le fruit de son travail! Vous allez donc compiler votre code.\\
Visual Studio vous permet de compiler très intuitivement et facilement si vous ne souhaitez pas d'options particulières. D'ailleurs, vous avez surement deviné que pour compiler il faut utiliser la petite flèche verte dans la barre d'outils! Eh bien ce n'est pas tout à fait exact, ce bouton fera deux choses pour vous: la compilation et l'exécution.
Ainsi, si vous cliquez sur ce bouton, vous verrez au bout d'un moment votre console affichant "Hello World". Il est toutefois possible de compiler SANS exécuter le programme automatiquement.
Pour cela, choisissez le menu "Générer -- Générer la solution". Compiler sans exécuter est utile si vous souhaitez juste voir les messages d'erreurs, ou vérifier que tout compile.\\
Toutefois, utiliser les boutons devient rapidement pénible. Il existe donc deux raccourcis: F5 pour compiler+exécuter et F6 pour compiler uniquement.\\
Faites donc un F5 et regardez le résultat!
\subsection {Première Fonction}
Votre première fonction va écrire Hello World! sur la sortie. Vous devrez
la placer dans le fichier Program.cs en dehors de la fonction Main(), mais à intérieur de la classe (délimitée par des accolades \{ \}) Program.
Votre fonction ressemblera à la fonction Main actuelle, mais sans arguments.\\
Toute fonction a un type de retour: int, float, string, etc... Toutefois, une fonction n'est pas toujours obligée de renvoyer quelque-chose!
Elle peut donc renvoyer 'rien', c'est à dire 'void'. Une telle fonction est usuellement appelée "procédure".\\
Lorsque vous déclarez une fonction, vous lui donnez donc: des attributs, un type, un nom, des paramètres, un contenu (le code exécuté lorsqu'elle est appelée).
Pour le moment, nous ne ferons pas attention aux attributs et mettrons 'static' qui signifie que cette fonction peut être appelée sans avoir à créer un objet.\\
Voici donc, à quoi devrait ressembler votre procédure HelloWorld():
\begin{lstlisting}
using System;
using System.Collections.Generic;
using System.Linq;
using System.Text;  

namespace ConsoleApplication1
{
    class Program
    {
        static void Main(string[] args)
        {
        }

        static void HelloWorld()
        {
            Console.WriteLine("Hello world");
            Console.ReadLine();
      }
   }
}

\end{lstlisting}
Maintenant que nous avons codé la fonction il faut l'appeler, c'est dans
le Main que vous allez le faire. Celui-ci va donc contenir :
\begin{lstlisting}
static void Main(string[] args)
       {
            HelloWorld();
        }
\end{lstlisting}
Bravo, vous êtes maintenant prêts à coder des fonctions plus compliquées !

\section{Factorielle}
\subsection{Définition de Factorielle}
Pour une gestion plus simple des cas d’erreurs nous allons imposer une
définition de factorielle spéciale qui vaut 1 pour des nombres négatifs.\\

\begin{center}
\begin{math}
n! = \left\{
    \begin{tabular}{ccc}
        1 & si & n <= 1\\\\
        n * (n - 1)! & si & n > 1
    \end{tabular}
\right. 
\end{math}
\end{center}


Afin de pouvoir tester votre code, voici quelques exemples.\\

\begin{center}
\begin{tabular}{|c|c|c|c|c|c|c|c|c|}
\hline
\textbf{n} & -42 & -1 & 0 & 1 & 2 & 3 & 5 & 10\\
\hline
\textbf{n!} & 1 & 1 & 1 & 1 & 2 & 6 & 120 & 3628800\\
\hline
\end{tabular}
\end{center}


\helpbox{Aide}{lorsque vous aurez à choisir des exemples par vous même par
la suite, veillez à bien prendre en compte tous les cas particuliers (et à les
implémenter) ! Ici il ne faut pas oublier les nombres négatifs, ni 0, ni 1.}
\subsection{Version Recursive}
Nous allons débuter par une version récursive de la fonction qui à déjà été
vue en \caml{}. Il s’agit d’écrire le code C\# qui va retranscrire à la lettre la
formule mathématique.
\begin{lstlisting}
static int fact(int n)
{
}
\end{lstlisting}
\emph{Syntaxe: } Pour réaliser des alternatives vous devez utiliser la structure
\begin{lstlisting}
if (Condition)
{
Action1();
}
else
{
Action2();
}
\end{lstlisting}
Si la Condition est égale à vrai alors l’Action1 est effectuée, sinon c’est
l’Action2 qui l’est.
\subsection{Version Impérative - While}
Nous allons transcrire cette fonction en impératif. Nous allons utiliser des boucles
pour simuler les appels récursifs. Il existe plusieurs types de boucles mais la
plus simple pour commencer reste la boucle While.
\begin{lstlisting}
static int fact_while(int n)
{
}
\end{lstlisting}
\emph{Syntaxe: } La boucle while s’utilise de cette façon :
\begin{lstlisting}
while (Condition) 
{
Action;
}
\end{lstlisting}
Tant que la Condition est vérifiée, alors on effectue l’Action. Avec la Condition
étant une expression de type bool.\\
Par exemple le code suivant va afficher
les nombres de i à 0.
\begin{lstlisting}
while (i>=0) 
{
      Console.WriteLine(i);
      i--; // ou bien i = i - 1;
}
\end{lstlisting}
\subsection{Version Imperative - For}
Maintenant que vous avez réussi à retranscrire grâce à la boucle While la
fonction factorielle, nous allons utiliser la boucle For. Lorsque l’on connait
le nombre d’itérations à l’avance elle est plus pratique à utiliser.
\begin{lstlisting}
static int fact_for(int n)
{
}
\end{lstlisting}

\emph{Syntaxe: } La boucle For s’utilise de cette façon :
\begin{lstlisting}
for(int i=0; i < 10; i++)
{
      action;
}
\end{lstlisting}
Pour comprendre son fonctionnement, on peut la réécrire avec une boucle
While. Vous remarquerez la concision de la boucle for.
\begin{lstlisting}
int i = 0;
while (i < 10) 
{
      action;
      i++; // ou bien i = i + 1;
}
\end{lstlisting}
\section{Point 2D}
\subsection{Type - Structure}
Nous allons apprendre à créer un nouveau type : un vecteur mathématique.
Il comporte deux coordonnées représentées par deux champs dans notre
structure.
\begin{lstlisting}
struct Point2D
    {
        public int x;
        public int y;
    }
\end{lstlisting}
Vous pouvez placer cette définition à deux endroits: avant la déclaration de la classe Program, ou
dans un fichier séparé. Le plus propre est encore de le faire dans un fichier séparé.\\
Si toutefois, vous souhaitez le mettre dans le même fichier que votre Main, vous devrez le placer dans la partie "namespace", 
mais avant la déclaration "class Program". Vous obtiendrez ceci:
\begin{lstlisting}
using System;
using System.Collections.Generic;
using System.Linq;
using System.Text;  

namespace ConsoleApplication1
{

    struct Point2D
    {
        public int x;
        public int y;
    }

    class Program
    {
        static void Main(string[] args)
        { /* du code ici */   }

        // d'autres fonctions ici 
    }

}

\end{lstlisting}
\emph{Utilisation: } Son utilisation est très simple. Il vous suffit de déclarer une
variable du type \textbf{Point2D}. Ensuite pour accéder à chacun des champs il vous
suffit de mettre un \textbf{.} devant son identifiant. Remarquez l'attribut "public" devant chaque définition de variable
de la structure \textbf{Point2D}. Cela signifie que vous donnez à tout votre code le droit de lire et modifier les champs (i-e x et y)
de la structure. En général, c'est une pratique à éviter car cela brise certains principes liés à la Programmation Orientée Objet.
Toutefois, ici notre structure est très simple, donc pour plus de simplicité, nous conserverons l'attribut public.
\begin{lstlisting}
Point2D pt;
pt.x = 0;
pt.x = pt.x + 1; // ou bien pt.x++
Console.WriteLine(pt.x);
// Affiche 1
\end{lstlisting}
\subsection{Afficher un Point2D}
La première chose à faire avant de coder quoi que ce soit est de s’assurer
que l’on peut afficher ce que l’on est en train de faire. Nous allons donc réaliser
une fonction qui prend un \textbf{Point2D} en paramètre, par exemple ${x = 12 \choose y = −5}$,
et qui affiche dans la console:\\

\begin{verbatim}
x=12 ; y=-5
\end{verbatim}

\begin{lstlisting}
static void print_pt2d(Point2D p)
{
}
\end{lstlisting}
\warnbox{Attention}{lorsque l’on vous demande d’écrire quelque chose dans la
sortie vous devez respecter le format à la lettre ! Pas d’espace autour du
‘=‘, les deux sur une même ligne ... Lorsque vous serez corrigés par une
moulinette, la moindre faute de syntaxe vous vaudra 0.}
\subsection{Addition}
Passons aux choses sérieuses. Vous devez coder l’addition de deux vecteurs,
inutile de vous donner la formule ...
\begin{lstlisting}
static Point2D add_pt2d(Point2D pa, Point2D pb)
{
}
\end{lstlisting}
\subsection{Opposé}
En prévision de la fonction de soustraction, nous allons coder une fonction
qui donne l’opposé d’un Point2D, c’est à dire l’opposé de tous les champs.
Par exemple:
\begin{center}
\begin{math}
-{x = 2 \choose y = -5} = {x = -2 \choose y = 5}
\end{math}
\end{center}
\begin{lstlisting}
static Point2D opp_pt2d(Point2D pa)
{
}
\end{lstlisting}
\subsection{Opposé par référence}
Il faut savoir que lorsque l’on retourne un type structuré par une fonction,
tous les champs sont recopiés dans une nouvelle variable. Même si dans cet
exemple ce n’est pas important, lorsque ces opérations sont effectuées des
millions de fois par seconde, toute optimisation, même aussi mineure que
celle-ci, est bonne à prendre.
La méthode pour palier ce problème est le passage par référence. En effet,
la variable est directement modifiée et non copiée. Par contre il n’est plus
possible d’utiliser l’ancienne variable.

\begin{lstlisting}
static void opp_pt2d(ref Point2D pa)
{
}
\end{lstlisting}
\subsection{Soustraction avec Opposé}
Maintenant que vous avez les deux fonctions Addition et Opposé, il va
faloir les combiner afin de coder la fonction Soustraction.
\begin{lstlisting}
static Point2D sub_pt2d(Point2D pa, Point2D pb)
{
}
\end{lstlisting}
\subsection{Distance}
Il est possible à partir de vecteurs d’obtenir d’autres types que des vecteurs.
Par exemple on peut calculer la distance entre deux vecteurs qui est
un réel.\\

\begin{center}
\begin{math}
dist({x1 \choose y1} , {x2 \choose y2}) = \sqrt{(x2 - x1)^{2} + (y2 - y1)^{2}}
\end{math}
\end{center}


\begin{lstlisting}
static float dist_pt2d(Point2D pa, Point2D pb)
{
}
\end{lstlisting}	
\helpbox{Remarque}{il existe plusieurs représentations des nombres réels en C\#.
Les plus couramment utilisés sont les float. Les double permettent de stocker de plus grandes valeurs.}\\
\helpbox{Aide}{Vous devrez utiliser la fonction racine carrée suivante : \textbf{Math.sqrt()}}
\section{Bonus}
Bravo, vous avez terminé le TP guidé, voici quelques exercices bonus pour
parfaire vos notions de C\#.
\subsection{Un générateur de particules!}
% end of document
\end{document}
